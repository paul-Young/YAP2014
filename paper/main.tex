\documentclass[aps,prl,superscriptaddress,12pt]{revtex4-1}
	\usepackage{graphicx}  % needed for figures
	\usepackage{fancyhdr} % needed for team number and page number
	\usepackage{amsmath}
	\usepackage{comment}
	% configure desired header format
	\pagestyle{fancy}
	\fancyhf{}
	\lhead{ Team \#  }
	\rhead{ Page \thepage\text{ }of \pageref{LastPage} }

\begin{document}

	\title{Macroscopic and Microscopic Assessment of Traffic Rules }
	%\author{Abigail Chua}
	%	\affiliation{Denison University, Granville, OH, 43023, USA}
	%\author{Yubo Yang}
	%	\affiliation{Denison University, Granville, OH, 43023, USA}
	%\author{Yifu Zhao}
	%	\affiliation{Denison University, Granville, OH, 43023, USA}
		
	\begin{abstract}
		In countries where vehicles travel on the right, vehicles on the freeway obey the keep-right-except-to-pass rule. This means that vehicles travel in the right-most lane of a highway unless they wish to overtake a car. To construct a simple and effective model, we consider a two-lane freeway with no on/off ramp. To describe the performance of the rule under varying traffic conditions, we investigated the flow of traffic, road safety and the ability of each lane to recover from a traffic jam. To simplify our models, we are only considering a stretch of straight two-lane freeway with no on/off ramps. This means that the total number of cars is constant and the only sources or sinks of cars are the two lanes. 
The flow of the traffic through each lane is modeled through a macroscopic lens. That is, we consider all cars to have uniform length l and maintain a uniform bumper-to-bumper distance d from their neighbors. Since the overtaking of other cars causes a small perturbation in traffic flow that eventually relaxes to the equilibrium velocity, we assume that all cars travel at this speed, ve(p). Because the highway is not always completely congested, there will be extra free space in between cars in addition to the safe following distance. We treat this space like it is occupied with an invisible vehicle density po. Combining these factors, we have the equation: Ve(p)=(1/(p1+p1)-l)/t
There are several interesting features which arise from this model. Namely, the speed-density relationship imposes a natural speed limit and jamming density on the road. More interestingly, however, our model suggests that if there is little traffic, flow is maximized when there are no rules imposed. The employment of a rule actually hinders traffic flow. However, when there is a high traffic density, employing the keep-right rule does not significantly impact traffic flow.  
The governing equations for each road were adapted from a hybrid of car-following theory and the classic LWR kinematics wave model (R.Jian and QS Wu) by incorporating the aforementioned speed-density function. We impose a periodic boundary and initial condition that represents a traffic disturbance in a lane. In accordance with our analysis of the speed-density function, when there is light traffic, a traffic disturbance in lane 1 is most quickly resolved when no rules are imposed. Conversely, in heavy traffic conditions, the keep-right rule helps to resolve the disturbance. 

	\end{abstract}
	
\maketitle

	\section{Interpretation of the Problem}

	\section{Introduction}
		Our writing style should follow guidelines such as \cite{science_writing}.

	\section{Model}
	
	\begin{table}
	\begin{tabular}{|c|c|c|} \hline
	quantity & variable & unit \\ \hline
	position on the freeway & x & m/s \\ \hline
	traffic density on lane 1 & $\rho_1(x,t)$ & cars/m \\ \hline
	traffic density on lane 2 & $\rho_2(x,t)$ & cars/m \\ \hline
	average velocity on lane 1 & $v_1(x,t)$ & m/s \\ \hline
	average velocity on lane 2 & $v_2(x,t)$ & m/s \\ \hline
	\end{tabular}
	\caption{ variables used in the macroscopic model \label{tab:variables} }
	\end{table}	
	
	To simplify the model, we consider a stretch of straight two-lane freeway with no on/off ramp. To assess the performance of the given traffic rule, we would like to analyze its capacity for traffic flow under both heavy and light traffic conditions. Traffic flow rate is best captured in a macroscopic model with variables shown in Table \ref{tab:variables}. The total amount of traffic flow will then be determined by 
	
	\begin{align}
	& Q(x,t) = \rho_1(x,t)\cdot v_1(x,t)+\rho_2(x,t)\cdot v_2(x,t) &
	\end{align}
	
	We further assume that at equilibrium, there is some velocity-density relation. Given this velocity-density relation, the flow at any given junction of the freeway at a given time is completely determined by the local traffic densities $\rho_1$ and $\rho_2$. 
Kerner elt. al. proposed the following relationship in 2002. 
	\begin{align}
	& v_e = v_o\left( (\frac{1+e^{\rho/\rho_m-0.25}}{0.06})^{-1} - 3.76\times10^{-6} \right) & \label{eq:Kerner}
	\end{align}
	It captures the key 
	\begin{figure}
	\includegraphics[scale=1]{plot/Q_p1_p2}
	\caption{equilibrium traffic flow as a function of traffic densities using Kerner's velocity-density relation\label{fig:Q_p1_p2}}
	\end{figure}
	
	
	To derive such a relation, we consider all cars to have uniform length $l(m)$, travel with uniform velocity $v_e(\rho)$ and maintain uniform bumper-to-bumper distance $d(m)$ from their neighbors. At this equilibrium, each car will take up a total space of $d+l$ on one lane of the freeway. Therefore, the density of cars
	\begin{align}
	& \rho = \frac{1}{d+l} & \label{eq:follow}
	\end{align}
	Incorporating the two-second rule enforced by the New York Sate Department of Motor Vehicles \cite{science_writing}, $d=2v_e(\rho)$, equation (\ref{eq:follow}) can be solved to obtain and expression for $v_e(\rho)$
	\begin{align}
	& v_e(\rho) =  .5(\frac{1}{\rho}-l)& 
	\end{align}
	
	\pagebreak

	\section{Result}

	\section{Discussion}
	
\bibliographystyle{unsrt}
\bibliography{ref/ref}
\end{document}