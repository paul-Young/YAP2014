In countries where vehicles travel on the right, vehicles on the freeway obey the keep-right-except-to-pass rule. This means that vehicles travel in the right-most lane of a highway unless they wish to overtake a car. We present a simple and elegant model of a two-lane freeway with no on/off ramp. To describe the performance of the rule under varying traffic conditions, we investigated the flow of traffic, road safety and the ability of each lane to recover from a traffic jam. To simplify our models, we are only considering a stretch of straight two-lane freeway with no on/off ramps. \\
The flow of the traffic through each lane is modeled through a macroscopic lens. That is, we consider all cars to be identical and to travel at the equilibrium velocity, $v_e(\rho)$, if there is no disturbance. Thus, we describe the equilibrium velocity as $v_e(\rho)=(1/(\rho+\rho_o)-l)/t$. There are several interesting features which arise from this model. Namely, the speed-density relationship imposes a natural speed limit and jamming density on the road. Additionally, our model suggests that if there is little traffic, flow is maximized when there are no rules imposed. The employment of a rule actually hinders traffic flow. However, when there is a high traffic density, employing the keep-right rule does not significantly impact traffic flow.  \\
The governing equations for each road were adapted from a hybrid of car-following theory and the classic LWR kinematics wave model \cite{tang_2004} by incorporating the aforementioned speed-density function. We impose periodic boundary and initial conditions that represent a traffic disturbance in a lane. In accordance with our analysis of the speed-density function, when there is light traffic, a traffic disturbance in lane 1 is most quickly resolved when no rules are imposed. Conversely, in heavy traffic conditions, the keep-right rule helps to resolve the disturbance. \\
To analyze the safety of each rule, we used a similar microscopic model to determine the relationship between adjacent cars. This is built upon a discreet realization of the kinematic model to simulate and calculate the situation of overtaking in both ruled and unruled traffic. According to our model, when traffic is light, imposing the rule will not increase safety. However, when the traffic is heavy, employing the keep-right rule will reduce safety significantly. Combining our previous conclusions, it is clear that in light traffic, no rules should be imposed. However in heavy traffic, we highly encourage the employment of the keep-right rule as it will significantly increase safety without hindering flow. 
