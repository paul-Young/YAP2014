In countries where vehicles travel on the right, vehicles on the freeway obey the keep-right-except-to-pass rule. This means that vehicles travel in the right-most lane of a highway unless they wish to overtake a car. To construct a simple and effective model, we consider a two-lane freeway with no on/off ramp. To describe the performance of the rule under varying traffic conditions, we investigated the flow of traffic, road safety and the ability of each lane to recover from a traffic jam. To simplify our models, we are only considering a stretch of straight two-lane freeway with no on/off ramps. This means that the total number of cars is constant and the only sources or sinks of cars are the two lanes. 
The flow of the traffic through each lane is modeled through a macroscopic lens. That is, we consider all cars to have uniform length l and maintain a uniform bumper-to-bumper distance d from their neighbors. Since the overtaking of other cars causes a small perturbation in traffic flow that eventually relaxes to the equilibrium velocity, we assume that all cars travel at this speed, ve(p). Because the highway is not always completely congested, there will be extra free space in between cars in addition to the safe following distance. We treat this space like it is occupied with an invisible vehicle density po. Combining these factors, we have the equation: Ve(p)=(1/(p1+p1)-l)/t
There are several interesting features which arise from this model. Namely, the speed-density relationship imposes a natural speed limit and jamming density on the road. More interestingly, however, our model suggests that if there is little traffic, flow is maximized when there are no rules imposed. The employment of a rule actually hinders traffic flow. However, when there is a high traffic density, employing the keep-right rule does not significantly impact traffic flow.  
The governing equations for each road were adapted from a hybrid of car-following theory and the classic LWR kinematics wave model (R.Jian and QS Wu) by incorporating the aforementioned speed-density function. We impose a periodic boundary and initial condition that represents a traffic disturbance in a lane. In accordance with our analysis of the speed-density function, when there is light traffic, a traffic disturbance in lane 1 is most quickly resolved when no rules are imposed. Conversely, in heavy traffic conditions, the keep-right rule helps to resolve the disturbance. 
